\section{Simulation}
\label{sec:Simulation}

The simulation of the MICE trackers is designed as a module for the MAUS software framework.  The simulation makes use of the GEANT4 standard physics libraries to describe a particle motion's through the scintillating fibre material and uses various statistical models in describing the effects of electronic noise.  The trackers are simulated on a per fibre basis and layered into the fibre planes in the expected doublet layer as shown in figure~\ref{fig:DoubletLayer}. 

The MAUS framework invokes a beamline module to generate a simulated beam, according to pre-defined user parameters.  A particle incident upon a tracker fibre is stepped through, in accordance with the defined parameters. GEANT4 is invoked in each of these steps in determining the resulting momentum change of the particle and the magnitude of energy  deposited into the fibre.  These values are recorded individually, the current step used in determining where the next step is taken, and recorded before the next step is processed.  After a defined number of particles have been generated and stepped through the experiment the results are collected into a MICE spill and sent to the tracker MC module for processing.  

Each one of the incoming MC hits has a simple conversion factor applied which determines the raw number of photo electrons (NPE) generated.  A single interaction with a scintillating fibre is simulated by many steps through the material and the figure for any one step is not limited to integer values.  The raw NPE from every step through the fibre is summed and feed into the simulation of noise in the tracker electronics (see section \ref{subsec:Noise}).

The final result from this process, which includes the channel number, the final processed NPE, and timing information from GEANT4 are used to create a SciFiDigit.  This final result is as described in section~\ref{subsec:GeneralDataStructure}, and the output Digit from the MC is identical to that created from DAQ data.  After all hits have been processed the data is handed back to MAUS to be processed by the tracker reconstruction modules.

  \subsection{Noise}
  \label{subsec:Noise}
  The MC noise simulation consist of two modules, false signal due to thermally excited electrons within the VLPC cassettes and a smearing due to random noise in the tracker electronics.  This is in addition to noise introduced by particle decays handled outside of the tracker MC by the GEANT4 simulation.
  
  Simulation of the thermally excited electrons is performed before the smearing simulation.  The dark count is an uncorrelated process that is selected to occur with a magnitude of 1.0 PE in 1.5$\%$ of a data taking window.  Actual rate is determined by the setting of the voltage bias in the VLPC cassettes which as a direct effect on the final signal size.  The process is described as a Poisson distribution.  Studies are under way to understand this effect in each cassette. 
  
  The results from the GEANT4 physics simulation and the Poisson dark count simulation are combined and smeared to determine the final NPE signal.  The signal from the GEANT4 simulation can take any value, however, it is unreasonable to expect anything other than an integer number of photons, as such this incoming signal is changed to its nearest integer value. The value is smeared as described by a Gaussian with a sigma derived from a study of carried out in May 2012 of a single station in beam.
  
  The smeared result is then fed through a process that simulates the effects of the analogue to digital converters (ADCs).  This process serves to chop the information up into bins of $2^8$ discreet values.  The exact values of these bins is determined from the tracker calibrations and varies with the with the channel placement into the electronics.  Overflows are equivalent to the maximum signal.  