\section{The MAUS Framework}
\label{sec:MAUS}
The tracker software is part of the MICE software framework, known as MAUS (MICE Analysis User Software)~\cite{MausPaper}. MAUS is used to perform Monte Carlo simulation and both online and offline data reconstruction. It is built using a combination of C++ and Python, with C++ being used predominantely in the code used for more processor intensive tasks (the ``backend'') and Python being used more in the code presented to the user (the``frontend'').  Simulation is supported by GEANT4~\cite{GEANT4}, and analysis with ROOT~\cite{ROOT}.  The input and output data formats can be either ROOT or JavaScript Object Notation (JSON) files (with ROOT being the standard). MAUS also reads in the custom binary format written by the MICE data acquisition system (DAQ). 

MAUS is controlled by the user using a top level Python script, together with a configuration file.  The structure at the top level is set out in a funcional-coding manner, with different modules being loaded depending on the task at hand.  The modules come in four types: Input; Output; Map; and Reduce.  Input modules provide the initial data to MAUS, from a data file, or from the DAQ. Maps perform most of the simulation and analysis work and may be processed in parallel across multiple nodes.  Reducers are used to display output, such as for online reconstruction plots, and are capable of accumulating data sent from maps over multiple spills, but must only be run in a single thread. Output modules provide data persistency by saving to a file in a standard format.

The tracker software is called using four maps and a reducer. The maps cover: digitisation of Monte Carlo data; digitisation of real DAQ data; the addition of noise to Monte Carlo data; and reconstruction. The reducer provides event plots and run information.  The modules contain little code themselves but instead call backend C++ classes.